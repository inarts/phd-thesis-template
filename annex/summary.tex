% !TEX root =  ../thesis-ms.tex
\setlength\parindent{0pt}
\setlength\parskip{.6em}
\linespread{1}
\cleardoublepage
~\vfill
\hfill\begin{minipage}{0.5\textwidth}
{\scriptsize
\textbf{\sffamily Colophon}\vskip 1em
Cette thèse a été composée sous \XeLaTeX, en utilisant Adobe Utopia pour le texte et Adobe Myriad Pro pour les éléments de titraille. Première impression le 2 septembre 2011, à partir de la version du \today, tirée à 5 exemplaires à l'Université Montpellier II.\vskip 1em
Le document final comporte \pageref{LastPage} pages dont \pageref{endpage} pour le texte principal. La bibliographie contient 391 références.\vskip 1em
Citation: \textsc{Poisot, T.} (2011) ``Spécialisation et structure dans des communautés en coévolution antagoniste''. Thèse de doctorat, Université Montpellier 2. \pageref{LastPage} pages.\\\hrule

}
\end{minipage}
\small
\clearpage\linespread{1}\newgeometry{inner=3.2cm,outer=2.5cm,top=2.5cm,bottom=2.5cm}%

\noindent\textbf{\sffamily  Spécialisation et structure dans des communautés en coévolution antagoniste}

{La spécialisation écologique est l'un des processus les plus complexes auxquels l'écologie évolutive s'intéresse, du fait des interactions complexes entre des processus écologiques et évolutifs. Malgré l'intérêt généré par cette notion depuis plusieurs décennies, nous ne disposons pas encore d'un cadre conceptuel suffisamment global pour intégrer les questionnements des différentes disciplines dans lesquelles la spécialisation est importante. Dans cette thèse, nous nous intéressons à la spécialisation en tant que processus, et au patron de spécificité qui en résulte, dans les systèmes en coévolution antagoniste. Dans une première partie, en synthétisant la littérature existante, nous établissons un cadre conceptuel intégratif pour étudier l'évolution de la spécialisation, et proposons une définition générale de ce trait. Sur la base de cette définition, nous évaluons les performances de différentes mesures de spécificité, et montrons que le \emph{Paired Differences Index} développé pendant cette thèse est une mesure à la fois robuste et informative. Dans une seconde partie, nous nous intéressons aux mécanismes écologiques et évolutifs qui provoquent l'établissement d'un patron de spécificité dans des systèmes antagonistes. L'innovation principale de nos travaux est de prendre en compte de manière explicite la dynamique des ressources, en étudiant d'abord les effets de variations dans leur disponibilité sur le niveau apparent de spécificité de parasites, puis en modélisant les effets de variations temporelles de l'apport en ressource. Ces résultats nous permettent de mieux comprendre comment des variations de productivité influent sur le processus coévolutif, et les conséquences sur la structure des communautés. Enfin, nous ouvrons ces résultats sur la question de la coexistence entre spécialistes et généralistes dans des communautés microbiennes naturelles, et montrons que la plupart des spécialistes, malgré une performance inférieure à celle des généralistes, se maintiennent en exploitant les niches laissées ouvertes par les autres espèces.

\noindent\textbf{Mot-clés:} spécialisation, écologie évolutive, coévolution antagoniste, niche, réseaux trophiques, \emph{Pseudomonas fluorescens}}
\vfill
\hrule
\vfill
\noindent\textbf{\sffamily Specialization and structure in antagonistically coevolving communities}

{Why specialization evolves is one of the most puzzling questions in evolutionary ecology, due to the complexity of interactions between ecological and evolutionary processes. In spite of the sustained interest in specialization over the last decades, we still lack an overarching framework. In this dissertation, we focus on specialization as a process, and on specificity as the resulting pattern, in antagonistically coevolving systems. In the first part, we synthesize the existing literature to establish a conceptual framework for investigating the evolution of specialization. We evaluate the performances of several measures of specificity, and develop a new measure -- the Paired Differences Index. We show it to be both robust and informative. In the second part, we focus on the ecological and evolutionary mechanisms that trigger the emergence of patterns in specificity in antagonistic systems. The main new feature of this work is that we explicitly account for resource dynamics, by first studying the consequences of variations in abundance on estimated specificity of parasites, and then by modeling the effect of seasonal resource dynamics on a coevolving community. These results enable us to better characterize how changes in productivity affect the coevolutionary process, and the consequences for community structure. Finally, we explore the question of the coexistence between specialists and generalists in natural microbial communities, and show that specialists can persist despite their comparatively lower performance, by exploiting open niches.

\noindent\textbf{Keywords:} specialization, evolutionary ecology, antagonistic coevolution, niche, food webs, \emph{Pseudomonas fluorescens}}

\vfill
\hrule

\noindent{\singlespacing\scriptsize%
Cette thèse a été préparée sous la direction de Michael E. \textsc{Hochberg}, à l'Institut des Sciences de l'\'Evolution, unité mixte de recherche 5554 du \textsc{cnrs} et de l'Université Montpellier II (Place Eugène Bataillon, \textsc{cc 065}, 34095 Montpellier), dans le groupe d'\'Evolution et \'Ecologie des Communautés.
}